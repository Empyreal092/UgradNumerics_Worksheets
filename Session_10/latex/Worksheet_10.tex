\documentclass[11pt,letterpaper]{article}
\usepackage[utf8]{inputenc}
\usepackage[left=1in,right=1in,top=1in,bottom=1in]{geometry}
\usepackage{amsfonts,amsmath}
\usepackage{graphicx,float}
% -----------------------------------
\usepackage{hyperref}
\hypersetup{%
  colorlinks=true,
  linkcolor=blue,
  citecolor=blue,
  urlcolor=blue,
  linkbordercolor={0 0 1}
}
% -----------------------------------
\usepackage[style=authoryear-icomp,backend=biber]{biblatex}
\addbibresource{citation.bib}
% -----------------------------------
\usepackage{fancyhdr}
\newcommand\course{MATH-UA.0252\\Numerical Analysis}
\newcommand\hwnumber{10}                  % <-- homework number
\newcommand\NetIDa{Ryan Sh\`iji\'e D\`u} 
\newcommand\NetIDb{November 17th, 2023}
\pagestyle{fancyplain}
\headheight 35pt
\lhead{\NetIDa\\\NetIDb}
\chead{\textbf{\Large Worksheet \hwnumber}}
\rhead{\course}
\lfoot{}
\cfoot{}
\rfoot{\small\thepage}
\headsep 1.5em
% -----------------------------------
\usepackage{titlesec}
\renewcommand\thesubsection{(\arabic{section}.\alph{subsection})}
\titleformat{\subsection}[runin]
        {\normalfont\bfseries}
        {\thesubsection}% the label and number
        {0.5em}% space between label/number and subsection title
        {}% formatting commands applied just to subsection title
        []% punctuation or other commands following subsection title
% -----------------------------------
\setlength{\parindent}{0.0in}
\setlength{\parskip}{0.1in}
% -----------------------------------
\newcommand{\de}{\mathrm{d}}
\newcommand{\DD}{\mathrm{D}}
\newcommand{\pe}{\partial}
\newcommand{\mcal}{\mathcal}
%\newcommand{\pdx}{\left|\frac{\partial}{\partial_x}\right|}

\newcommand{\dsp}{\displaystyle}

\newcommand{\norm}[1]{\left\Vert #1 \right\Vert}
%\newcommand{\mean}[1]{\left\langle #1 \right\rangle}
\newcommand{\mean}[1]{\overline{#1}}
\newcommand{\inner}[2]{\left\langle #1,#2\right\rangle}

\newcommand{\ve}[1]{\boldsymbol{#1}}

\newcommand{\thus}{\Rightarrow \quad }
\newcommand{\fff}{\iff\quad}
\newcommand{\qdt}[1]{\quad \mbox{#1} \quad}

\renewcommand{\Re}{\mathrm{Re}}
\renewcommand{\Im}{\mathrm{Im}}
\newcommand{\E}{\mathbb{E}}
\newcommand{\lap} {\nabla^2}
\renewcommand{\div}{\nabla\cdot}

\newcommand{\csch}{\text{csch}}
\newcommand{\sech}{\text{sech}}


\newcommand{\hot}{\text{h.o.t.}}

\newcommand{\ssp}{\left.\qquad\right.}

\newcommand{\var}{\text{var}}
\newcommand{\cov}{\text{cov}}

%%%%%%%%%%%%%%%%%%%%%%%%%%%%%%%%%%%%%%%%%%%%%%%%%%
\makeatletter
\newcommand*{\mint}[1]{%
  % #1: overlay symbol
  \mint@l{#1}{}%
}
\newcommand*{\mint@l}[2]{%
  % #1: overlay symbol
  % #2: limits
  \@ifnextchar\limits{%
    \mint@l{#1}%
  }{%
    \@ifnextchar\nolimits{%
      \mint@l{#1}%
    }{%
      \@ifnextchar\displaylimits{%
        \mint@l{#1}%
      }{%
        \mint@s{#2}{#1}%
      }%
    }%
  }%
}
\newcommand*{\mint@s}[2]{%
  % #1: limits
  % #2: overlay symbol
  \@ifnextchar_{%
    \mint@sub{#1}{#2}%
  }{%
    \@ifnextchar^{%
      \mint@sup{#1}{#2}%
    }{%
      \mint@{#1}{#2}{}{}%
    }%
  }%
}
\def\mint@sub#1#2_#3{%
  \@ifnextchar^{%
    \mint@sub@sup{#1}{#2}{#3}%
  }{%
    \mint@{#1}{#2}{#3}{}%
  }%
}
\def\mint@sup#1#2^#3{%
  \@ifnextchar_{%
    \mint@sup@sub{#1}{#2}{#3}%
  }{%
    \mint@{#1}{#2}{}{#3}%
  }%
}
\def\mint@sub@sup#1#2#3^#4{%
  \mint@{#1}{#2}{#3}{#4}%
}
\def\mint@sup@sub#1#2#3_#4{%
  \mint@{#1}{#2}{#4}{#3}%
}
\newcommand*{\mint@}[4]{%
  % #1: \limits, \nolimits, \displaylimits
  % #2: overlay symbol: -, =, ...
  % #3: subscript
  % #4: superscript
  \mathop{}%
  \mkern-\thinmuskip
  \mathchoice{%
    \mint@@{#1}{#2}{#3}{#4}%
        \displaystyle\textstyle\scriptstyle
  }{%
    \mint@@{#1}{#2}{#3}{#4}%
        \textstyle\scriptstyle\scriptstyle
  }{%
    \mint@@{#1}{#2}{#3}{#4}%
        \scriptstyle\scriptscriptstyle\scriptscriptstyle
  }{%
    \mint@@{#1}{#2}{#3}{#4}%
        \scriptscriptstyle\scriptscriptstyle\scriptscriptstyle
  }%
  \mkern-\thinmuskip
  \int#1%
  \ifx\\#3\\\else_{#3}\fi
  \ifx\\#4\\\else^{#4}\fi  
}
\newcommand*{\mint@@}[7]{%
  % #1: limits
  % #2: overlay symbol
  % #3: subscript
  % #4: superscript
  % #5: math style
  % #6: math style for overlay symbol
  % #7: math style for subscript/superscript
  \begingroup
    \sbox0{$#5\int\m@th$}%
    \sbox2{$#5\int_{}\m@th$}%
    \dimen2=\wd0 %
    % => \dimen2 = width of \int
    \let\mint@limits=#1\relax
    \ifx\mint@limits\relax
      \sbox4{$#5\int_{\kern1sp}^{\kern1sp}\m@th$}%
      \ifdim\wd4>\wd2 %
        \let\mint@limits=\nolimits
      \else
        \let\mint@limits=\limits
      \fi
    \fi
    \ifx\mint@limits\displaylimits
      \ifx#5\displaystyle
        \let\mint@limits=\limits
      \fi
    \fi
    \ifx\mint@limits\limits
      \sbox0{$#7#3\m@th$}%
      \sbox2{$#7#4\m@th$}%
      \ifdim\wd0>\dimen2 %
        \dimen2=\wd0 %
      \fi
      \ifdim\wd2>\dimen2 %
        \dimen2=\wd2 %
      \fi
    \fi
    \rlap{%
      $#5%
        \vcenter{%
          \hbox to\dimen2{%
            \hss
            $#6{#2}\m@th$%
            \hss
          }%
        }%
      $%
    }%
  \endgroup
}

\begin{document} 

\section{Polynomial interpolation and linear algebra}
In lecture you have learned the theorem which states:

For $n\geq 1$ and distinct $n+1$ data pairs $(x_0,y_0),\dots,(x_n,y_n)$ there exists a unique $p_n(x)\in P_n$, an $n$-th order polynomial such that $p_n(x_i) = y_i$ for $i = 0,\dots,n$. 

We will try to prove this using linear algebra.

\subsection{}
We can write an $n$-th order polynomial as
\begin{align*}
    p_n(x) = a_0 + a_1x + \dots + a_nx^n.
\end{align*}
This gives us $n+1$ free variables to solve. Frame the problem of finding $p_n(x)$ s.t. $p_n(x_i) = y_i$ for $i = 0,\dots,n$ as a matrix problem $X\ve a = \ve y$.

\subsection{}
Show that since $x_i\neq x_j$ for all $i,j$, we have the matrix $X$ we constructed has full rank. 

\subsection{}
Think about the uniqueness and existence claim in the theorem in linear algebra language. 

\section{Interpolation basics}
\begin{itemize}
\item True or False? For the nodes $x_0=0, x_1=1, x_2= 2$, the
  Lagrange interpolation polynomial $L_0(x)$ is $-x^2 + 1$.
\item True or False? We compute the Hermite interpolant with 3 distinct nodes of
  a function $f$ that is a polynomial of degree $4$. Then this
  Hermite interpolant is identical to $f$. (In short: Hermite
  interpolation with 3 nodes is exact for polynomials of degree 4.)
\item True or False? Hermite interpolation with 4 distinct nodes is exact for
  polynomials of degree 6.
\item True or false: Let $p_n$ be the Lagrange interpolant to a
function $f$ with $n+1$ interpolation points, and $e_n(x) =
|p_n(x) - f(x)|$.  The interpolation error $\|e_n\|_\infty$ {\em
always} gets arbitrarily small for large $n$, i.e.,
$\|e_n\|_\infty \rightarrow 0$ as $n \rightarrow \infty$.
\end{itemize}

\section{Lagrange interpolation polynomial example}
Let $x_0,\ldots, x_n$ be distinct interpolation nodes, and let
  $$
  p_n(x) = \sum^n_{k=0}L_{k}(x)(x_k)^j,
  $$
where $j$ is an integer and $n \geq j>0$. What is the $p_n(x)$ function? What are the values of $p_n(0)$ and $p_n(1)$?

\section{Hermite interpolation polynomial example}
Recall that the Hermite interpolation of a function $f$ at the points $x_0,x_1,x_2$ has the form 
$$p(x) = \sum_{j=0}^2H_j(x)f(x_j)
+ \sum_{j=0}^2K_j(x)f'(x_j).$$ 
  
\subsection{}
Show that the polynomial
$$ -\frac{1}{\pi}x^2 + x$$ 
is the Hermite interpolation polynomial of $f(x):=\sin(x)$ based on the nodes $x_0=0$, $x_1=\pi$.
  
\subsection{}
Show that the polynomial $K_2(x)$ in
this representation for $x_0=0,x_1=1,x_2=2$ is given by
$$
\frac{1}{4}x^5 - x^4 + \frac{5}{4}x^3 - \frac{1}{2}x^2.
$$

\section{Error bound for interpolation}
This requires MATLAB or Python: we interpolate the function $f:[0,1]\to\mathbb R$ defined by
  $f(x) = \exp(3x)$ using the nodes $x_i=i/2$, $i=0,1,2$ by a
  quadratic polynomial $p_2\in \boldsymbol P_2$.
 Compare the exact interpolation error $E_f(x):=f(x)-p_2(x)$ at
 $x=3/4$ with the estimate
  \begin{equation*}
  |E_f(x)|\le \frac{M_{n+1}}{(n+1)!}|\pi_{n+1}(x)|,
  \end{equation*}
  where $M_{n+1} = \max_{z\in [0,1]}|f^{(n+1)}(z)|$, $f^{(n+1)}$ is
  the $(n+1)$st derivative of $f$, and
  $\pi_{n+1}(x)=(x-x_0)(x-x_1)(x-x_2)$.

\vfill
\printbibliography

\end{document}